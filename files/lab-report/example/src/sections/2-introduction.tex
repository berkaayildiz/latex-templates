\section{Introduction}

A 7-segment display consists of seven LEDs arranged in a rectangular fashion. Each of the seven LEDs is called a segment because when illuminated the segment forms part of a numerical digit to be displayed. Each one of the seven LEDs in the display is given a positional segment with one of its connection pins being brought straight out of the rectangular plastic package. These individually LED pins are labelled from a through to g representing each individual LED. The other LED pins are connected together and wired to form a common pin.\footnote{Answer to "What is a 7-segment display and how it works?"}

The displays common pin is generally used to identify which type of 7-segment display it is. As each LED has two connecting pins, one called the “Anode” and the other called the “Cathode”, there are therefore two types of LED 7-segment display called: Common Cathode (CC) and Common Anode (CA).
The difference between the two displays, is that the common cathode has all the cathodes of the 7-segments connected directly together and the common anode has all the anodes of the 7-segments connected together and is illuminated as follows. In a common cathode display, all LED segment cathodes connect to ground (logic "0"), and each segment lights up with a "HIGH" (logic "1") signal and a current-limiting resistor applied to its anode terminals (a-g), while in a common anode display, all LED segment anodes connect to logic "1," and individual segments (a-g) light up with a ground (logic "0" or "LOW") signal applied to their cathodes through a suitable current-limiting resistor.\footnote{Answer to "How many types of 7-segment display are there and what sets them apart?"}
\cite{7-segment}

To use a 7-segment display with ease, a 7-segment display decoder is needed. The primary function of a 7-segment display decoder is to convert the input data into the appropriate signals that can drive the display. To achieve this, the decoder analyzes the input data and maps it to a specific combination of segment control signals. This mapping process can be implemented using combinational logic circuits. For example, consider a BCD to 7-segment display decoder. The BCD input consists of four binary digits, representing a decimal number between 0 and 9. The decoder’s task is to convert this BCD input into the appropriate segment control signals. This can be achieved by designing a combinational logic circuit that generates the correct combination of signals for each BCD input.\footnote{Answer to "Why do we need a decoder to use 7-segment displays?"} \cite{7-segment-decoder}
